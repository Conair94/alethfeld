% Alethfeld Generated LaTeX
% Lemma: L2-influence (Influence Independence)
% Graph Version: verified-2024
% Status: VERIFIED | Taint: CLEAN
% Generated by Alethfeld Orchestrator v4
% Rigor Setting: STRICTEST

\documentclass[11pt]{article}
\usepackage{amsmath,amssymb,amsthm}
\usepackage{braket}
\usepackage{geometry}
\usepackage{enumitem}
\usepackage{xcolor}

\geometry{margin=1in}

\theoremstyle{plain}
\newtheorem{lemma}{Lemma}
\newtheorem{claim}{Claim}
\theoremstyle{definition}
\newtheorem{definition}{Definition}
\newtheorem{assumption}{Assumption}

\newcommand{\Tr}{\operatorname{Tr}}

\title{Lemma L2: Influence Independence\\[0.5em]
\large Alethfeld Verified Proof}
\author{Alethfeld Proof Orchestrator v4}
\date{Verification Status: \textcolor{green!60!black}{\textbf{VERIFIED}} $\mid$ Taint: \textcolor{green!60!black}{\textbf{CLEAN}}\\[0.5em]
Rigor: \textcolor{blue!60!black}{\textbf{STRICTEST}}}

\begin{document}
\maketitle

\section*{Dependencies}

\begin{assumption}[A1: Product State QBF]\label{ass:A1}
Let $U = I - 2\ket{\psi}\bra{\psi}$ be a rank-1 quantum Boolean function where $\ket{\psi} = \bigotimes_{k=1}^n \ket{\phi_k}$ is a product state.
\end{assumption}

\begin{definition}[D1: Bloch Vector]\label{def:D1}
For each qubit $k \in \{1, \ldots, n\}$, the Bloch vector $\vec{r}_k = (x_k, y_k, z_k)$ satisfies the unit sphere constraint:
\[
x_k^2 + y_k^2 + z_k^2 = 1.
\]
\end{definition}

\begin{definition}[D2: Squared Bloch Components]\label{def:D2}
Define the squared Bloch components:
\[
q_k^{(0)} = 1, \quad q_k^{(1)} = x_k^2, \quad q_k^{(2)} = y_k^2, \quad q_k^{(3)} = z_k^2.
\]
\end{definition}

\begin{lemma}[L1: Fourier Coefficient Formula --- External Reference]\label{lem:L1}
For any multi-index $\alpha \in \{0,1,2,3\}^n$:
\[
\hat{U}(\alpha) = \delta_{\alpha,0} - 2^{1-n} \prod_{k=1}^n r_k^{(\alpha_k)}.
\]
\end{lemma}

\begin{lemma}[L1-Corollary: Probability Distribution]\label{lem:L1cor}
For $\alpha \neq 0$, the probability $p_\alpha = |\hat{U}(\alpha)|^2$ satisfies:
\[
p_\alpha = 2^{2-2n} \prod_{k=1}^n q_k^{(\alpha_k)}.
\]
\end{lemma}

\section*{Statement}

\begin{lemma}[L2: Influence Independence]\label{lem:L2}
For any rank-1 product state QBF on $n$ qubits:
\[
I(U) = n \cdot 2^{1-n}.
\]
This value is independent of the choice of single-qubit states (Bloch vectors).
\end{lemma}

\section*{Proof}

\subsection*{Step 1: Definition of Influence}

\begin{claim}[1-step1]
The influence of qubit $j$ is:
\[
I_j = \sum_{\alpha: \alpha_j \neq 0} p_\alpha.
\]
\end{claim}

\begin{proof}
We establish this in substeps.

\textbf{(1a)} By definition, the influence $I_j$ measures how much the output of the function depends on qubit $j$.

\textbf{(1b)} For a classical Boolean function $f: \{0,1\}^n \to \{0,1\}$, the influence is defined as:
\[
I_j(f) = \Pr_x[f(x) \neq f(x^{\oplus j})]
\]
where $x^{\oplus j}$ denotes $x$ with bit $j$ flipped.

\textbf{(1c)} For quantum Boolean functions, this generalizes via the Pauli-Fourier expansion.
\begin{itemize}[leftmargin=2em]
\item[(1c.1)] The Pauli-Fourier expansion is $U = \sum_\alpha \hat{U}(\alpha) \sigma^\alpha$.
\item[(1c.2)] The index $\alpha_j \neq 0$ captures when qubit $j$ is acted upon non-trivially by $\sigma^{\alpha_j} \in \{\sigma_x, \sigma_y, \sigma_z\}$.
\end{itemize}

Therefore, $I_j = \sum_{\alpha: \alpha_j \neq 0} |\hat{U}(\alpha)|^2 = \sum_{\alpha: \alpha_j \neq 0} p_\alpha$.
\end{proof}

\subsection*{Step 2: Factorization of Partial Sum}

\begin{claim}[1-step2]
For each $\ell \in \{1, 2, 3\}$:
\[
\sum_{\alpha: \alpha_j = \ell} p_\alpha = 2^{2-2n} \cdot q_j^{(\ell)} \cdot \prod_{k \neq j} \sum_{m=0}^3 q_k^{(m)}.
\]
\end{claim}

\begin{proof}
\textbf{(2a)} From Lemma~\ref{lem:L1cor}, for $\alpha \neq 0$:
\[
p_\alpha = 2^{2-2n} \prod_{k=1}^n q_k^{(\alpha_k)}.
\]

\textbf{(2b)} Fixing $\alpha_j = \ell$, the product factors:
\begin{itemize}[leftmargin=2em]
\item[(2b.1)] Products over disjoint index sets factor: for $A \cap B = \emptyset$,
\[
\prod_{k \in A \cup B} f_k = \prod_{k \in A} f_k \cdot \prod_{k \in B} f_k.
\]
\item[(2b.2)] Setting $A = \{j\}$ and $B = \{1, \ldots, n\} \setminus \{j\}$:
\[
\prod_{k=1}^n q_k^{(\alpha_k)} = q_j^{(\ell)} \cdot \prod_{k \neq j} q_k^{(\alpha_k)}.
\]
\end{itemize}

\textbf{(2c)} Summing over all multi-indices with $\alpha_j = \ell$:
\begin{itemize}[leftmargin=2em]
\item[(2c.1)] Each index $\alpha_k$ for $k \neq j$ ranges independently over $\{0, 1, 2, 3\}$.
\item[(2c.2)] Products distribute over sums (Fubini for finite sums):
\[
\sum_{\alpha_1, \ldots, \alpha_{n-1}} \prod_k f_k(\alpha_k) = \prod_k \sum_{\alpha_k} f_k(\alpha_k).
\]
\end{itemize}

Combining:
\[
\sum_{\alpha: \alpha_j = \ell} p_\alpha = 2^{2-2n} \cdot q_j^{(\ell)} \cdot \prod_{k \neq j} \sum_{m=0}^3 q_k^{(m)}. \qedhere
\]
\end{proof}

\subsection*{Step 3: Unit Sphere Simplification}

\begin{claim}[1-step3]
Since $\sum_{m=0}^3 q_k^{(m)} = 2$, we have:
\[
\sum_{\alpha: \alpha_j = \ell} p_\alpha = 2^{1-n} q_j^{(\ell)}.
\]
\end{claim}

\begin{proof}
\textbf{(3a)} By Definition~\ref{def:D2}:
\begin{itemize}[leftmargin=2em]
\item[(3a.1)] Substituting $q_k^{(0)} = 1$, $q_k^{(1)} = x_k^2$, $q_k^{(2)} = y_k^2$, $q_k^{(3)} = z_k^2$:
\end{itemize}
\[
\sum_{m=0}^3 q_k^{(m)} = 1 + x_k^2 + y_k^2 + z_k^2.
\]

\textbf{(3b)} By Definition~\ref{def:D1} (unit sphere constraint):
\begin{itemize}[leftmargin=2em]
\item[(3b.1)] The Bloch vector satisfies $|\vec{r}_k|^2 = x_k^2 + y_k^2 + z_k^2 = 1$.
\end{itemize}
Therefore: $\sum_{m=0}^3 q_k^{(m)} = 1 + 1 = 2$.

\textbf{(3c)} Substituting into Step 2:
\begin{itemize}[leftmargin=2em]
\item[(3c.1)] The product $\prod_{k \neq j} 2 = 2^{n-1}$ since there are $n-1$ factors.
\item[(3c.2)] Exponent arithmetic: $(2-2n) + (n-1) = 2 - 2n + n - 1 = 1 - n$.
\end{itemize}
\[
\sum_{\alpha: \alpha_j = \ell} p_\alpha = 2^{2-2n} \cdot q_j^{(\ell)} \cdot 2^{n-1} = 2^{1-n} q_j^{(\ell)}. \qedhere
\]
\end{proof}

\subsection*{Step 4: Single-Qubit Influence}

\begin{claim}[1-step4]
For all $j \in \{1, \ldots, n\}$:
\[
I_j = 2^{1-n}
\]
independent of the choice of Bloch vector $\vec{r}_j$.
\end{claim}

\begin{proof}
\textbf{(4a)} From Step 1, partitioning by the value of $\alpha_j$:
\begin{itemize}[leftmargin=2em]
\item[(4a.1)] The condition $\alpha_j \neq 0$ is equivalent to $\alpha_j \in \{1, 2, 3\}$.
\end{itemize}
\[
I_j = \sum_{\alpha: \alpha_j \neq 0} p_\alpha = \sum_{\ell=1}^3 \sum_{\alpha: \alpha_j = \ell} p_\alpha.
\]

\textbf{(4b)} Applying Step 3:
\begin{itemize}[leftmargin=2em]
\item[(4b.1)] The constant $2^{1-n}$ factors out of the sum over $\ell$.
\end{itemize}
\[
I_j = \sum_{\ell=1}^3 2^{1-n} q_j^{(\ell)} = 2^{1-n} \sum_{\ell=1}^3 q_j^{(\ell)}.
\]

\textbf{(4c)} By Definitions~\ref{def:D1} and~\ref{def:D2}:
\begin{itemize}[leftmargin=2em]
\item[(4c.1)] From D2: $q_j^{(1)} + q_j^{(2)} + q_j^{(3)} = x_j^2 + y_j^2 + z_j^2$.
\item[(4c.2)] From D1: $x_j^2 + y_j^2 + z_j^2 = 1$ (unit sphere).
\end{itemize}
\[
I_j = 2^{1-n} \cdot 1 = 2^{1-n}. \qedhere
\]
\end{proof}

\subsection*{Conclusion}

\begin{proof}[Proof of Lemma~\ref{lem:L2}]
\textbf{(QED-a)} By definition, total influence is:
\begin{itemize}[leftmargin=2em]
\item[(a.1)] Total influence sums individual qubit influences over all $n$ qubits.
\end{itemize}
\[
I(U) = \sum_{j=1}^n I_j.
\]

\textbf{(QED-b)} From Step 4:
\begin{itemize}[leftmargin=2em]
\item[(b.1)] The value $2^{1-n}$ depends only on $n$, not on any Bloch vector.
\end{itemize}
Each $I_j = 2^{1-n}$, independent of $j$ and the choice of single-qubit states.

\textbf{(QED-c)} Therefore:
\begin{itemize}[leftmargin=2em]
\item[(c.1)] Summing the constant $2^{1-n}$ over $j \in \{1, \ldots, n\}$ gives $n \cdot 2^{1-n}$.
\item[(c.2)] This is the unique total influence for all rank-1 product state QBFs on $n$ qubits.
\end{itemize}
\[
I(U) = \sum_{j=1}^n 2^{1-n} = n \cdot 2^{1-n}. \qed
\]
\end{proof}

\section*{Remarks}

This is a remarkable result: the total influence $I(U) = n \cdot 2^{1-n}$ is \emph{completely independent} of the choice of single-qubit states. This universality arises because:
\begin{enumerate}
\item The unit sphere constraint forces $\sum_{\ell=1}^3 q_k^{(\ell)} = 1$ for every qubit.
\item This exact cancellation occurs for each qubit independently.
\item The only remaining dependence is on $n$, the number of qubits.
\end{enumerate}

This universality is what makes the entropy-influence ratio analysis tractable in the main theorem.

\vfill
\hrule
\vspace{0.5em}
\noindent\textbf{Alethfeld Verification Report}\\
\begin{tabular}{@{}ll@{}}
Graph ID: & L2-influence-verify-2024\\
Total Nodes: & 43 (5 original + 38 expanded)\\
Max Depth: & 3\\
Admitted Steps: & 0\\
Rigor Level: & STRICTEST\\
Taint Status: & \textcolor{green!60!black}{CLEAN}\\
\end{tabular}

\end{document}
