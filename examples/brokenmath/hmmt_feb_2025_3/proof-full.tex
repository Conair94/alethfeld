\documentclass[11pt,a4paper]{article}

% Mathematics packages
\usepackage{amsmath}
\usepackage{amsthm}
\usepackage{amssymb}
\usepackage{mathtools}

% Layout and formatting
\usepackage[margin=1in]{geometry}
\usepackage{enumitem}
\usepackage{booktabs}
\usepackage{array}
\usepackage{longtable}

% Theorem environments
\theoremstyle{plain}
\newtheorem{theorem}{Theorem}
\newtheorem{lemma}[theorem]{Lemma}
\newtheorem{proposition}[theorem]{Proposition}
\newtheorem{corollary}[theorem]{Corollary}

\theoremstyle{definition}
\newtheorem{definition}[theorem]{Definition}
\newtheorem{assumption}{Assumption}

\theoremstyle{remark}
\newtheorem{remark}[theorem]{Remark}
\newtheorem{claim}{Claim}

% Custom commands
\newcommand{\R}{\mathbb{R}}
\newcommand{\N}{\mathbb{N}}
\newcommand{\logb}[1]{\log_2\left(#1\right)}

\title{Minimum Value of $xyz$ Subject to Exponential-Logarithmic Constraints\\[0.5em]
\large HMMT February 2025, Problem 3}
\author{Formalized by Alethfeld Proof System}
\date{December 31, 2025}

\begin{document}

\maketitle

\begin{abstract}
We prove that given positive real numbers $x$, $y$, and $z$ satisfying the constraints
$x^{\log_2(yz)} = 2^8 \cdot 3^4$, $y^{\log_2(zx)} = 2^9 \cdot 3^6$, and $z^{\log_2(xy)} = 2^5 \cdot 3^{10}$,
the minimum possible value of $xyz$ is $\boxed{576}$.
The proof proceeds by transforming the system via logarithms, reducing to a constrained quadratic system,
analyzing all sign combinations in the solution space, and establishing uniqueness of the minimum
through monotonicity arguments.
\end{abstract}

\tableofcontents
\newpage

%% ============================================================
%% SECTION 1: Problem Statement and Setup
%% ============================================================
\section{Problem Statement and Setup}

\begin{theorem}[Main Result]\label{thm:main}
Given that $x, y, z \in \R^+$ such that
\begin{align}
x^{\log_2(yz)} &= 2^8 \cdot 3^4, \label{eq:constraint1}\\
y^{\log_2(zx)} &= 2^9 \cdot 3^6, \label{eq:constraint2}\\
z^{\log_2(xy)} &= 2^5 \cdot 3^{10}, \label{eq:constraint3}
\end{align}
the smallest possible value of $xyz$ is $576$.
\end{theorem}

We begin by establishing our notation and fundamental assumptions.

\begin{assumption}[Positivity]\label{as:positive}
$x, y, z \in \R^+$ (positive real numbers).
\end{assumption}

\begin{definition}[Logarithmic Variables]\label{def:logvars}
Define
\begin{equation}\label{eq:def-abc}
a = \log_2 x, \quad b = \log_2 y, \quad c = \log_2 z, \quad \alpha = \log_2 3.
\end{equation}
\end{definition}

\begin{remark}
Since $x, y, z > 0$, the quantities $a, b, c$ are well-defined real numbers.
The constant $\alpha = \log_2 3 \approx 1.585$ satisfies $1 < \alpha < 2$ since $2 < 3 < 4$.
\end{remark}

%% ============================================================
%% SECTION 2: Logarithmic Transformation
%% ============================================================
\section{Logarithmic Transformation}\label{sec:logtransform}

\begin{proposition}[Transformed System]\label{prop:transformed}
Under Definition~\ref{def:logvars}, the constraints \eqref{eq:constraint1}--\eqref{eq:constraint3} become:
\begin{align}
a(b+c) &= 8 + 4\alpha, \label{eq:trans1}\\
b(c+a) &= 9 + 6\alpha, \label{eq:trans2}\\
c(a+b) &= 5 + 10\alpha. \label{eq:trans3}
\end{align}
\end{proposition}

\begin{proof}
We derive equation~\eqref{eq:trans1} in detail; the others follow analogously.

\textbf{Step 1:} Apply $\log_2$ to both sides of \eqref{eq:constraint1}:
\begin{equation}\label{eq:step1}
\log_2\left(x^{\log_2(yz)}\right) = \log_2\left(2^8 \cdot 3^4\right).
\end{equation}

\textbf{Step 2:} Simplify the left-hand side using the power rule:
\begin{equation}\label{eq:step2}
\log_2\left(x^{\log_2(yz)}\right) = \log_2(yz) \cdot \log_2(x).
\end{equation}

\textbf{Step 3:} Apply the product rule to $\log_2(yz)$:
\begin{equation}\label{eq:step3}
\log_2(yz) = \log_2 y + \log_2 z = b + c.
\end{equation}

\textbf{Step 4:} Substitute Definition~\ref{def:logvars}:
\begin{equation}\label{eq:step4}
\text{LHS} = (b+c) \cdot a = a(b+c).
\end{equation}

\textbf{Step 5:} Simplify the right-hand side using the product rule:
\begin{equation}\label{eq:step5}
\log_2\left(2^8 \cdot 3^4\right) = \log_2(2^8) + \log_2(3^4).
\end{equation}

\textbf{Step 6:} Apply the power rule:
\begin{equation}\label{eq:step6}
\log_2(2^8) = 8, \quad \log_2(3^4) = 4\log_2 3 = 4\alpha.
\end{equation}

\textbf{Step 7:} Combine:
\begin{equation}\label{eq:step7}
\text{RHS} = 8 + 4\alpha.
\end{equation}

Therefore $a(b+c) = 8 + 4\alpha$.

\medskip
\textbf{For equation~\eqref{eq:trans2}:} Starting from $y^{\log_2(zx)} = 2^9 \cdot 3^6$:
\[
\log_2(zx) \cdot \log_2 y = \log_2(2^9 \cdot 3^6) \implies (c+a) \cdot b = 9 + 6\alpha.
\]

\textbf{For equation~\eqref{eq:trans3}:} Starting from $z^{\log_2(xy)} = 2^5 \cdot 3^{10}$:
\[
\log_2(xy) \cdot \log_2 z = \log_2(2^5 \cdot 3^{10}) \implies (a+b) \cdot c = 5 + 10\alpha.
\]
\end{proof}

%% ============================================================
%% SECTION 3: Reduction to Quadratic System
%% ============================================================
\section{Reduction to Quadratic System}\label{sec:quadratic}

\begin{definition}[Sum Variable]\label{def:sum}
Let $s = a + b + c$. Then $xyz = 2^a \cdot 2^b \cdot 2^c = 2^s$.
\end{definition}

Our goal becomes: minimize $s$ subject to equations \eqref{eq:trans1}--\eqref{eq:trans3}.

\begin{lemma}[Sum of Products]\label{lem:sumproducts}
Adding equations \eqref{eq:trans1}--\eqref{eq:trans3}:
\begin{equation}\label{eq:sum}
2(ab + bc + ca) = 22 + 20\alpha.
\end{equation}
\end{lemma}

\begin{proof}
\[
a(b+c) + b(c+a) + c(a+b) = (8+4\alpha) + (9+6\alpha) + (5+10\alpha) = 22 + 20\alpha.
\]
The left-hand side equals $ab + ac + bc + ba + ca + cb = 2(ab + bc + ca)$.
\end{proof}

\begin{proposition}[Quadratic Reformulation]\label{prop:quadratic}
Using $b + c = s - a$, etc., the system becomes:
\begin{align}
a^2 - sa + (8+4\alpha) &= 0, \label{eq:quad-a}\\
b^2 - sb + (9+6\alpha) &= 0, \label{eq:quad-b}\\
c^2 - sc + (5+10\alpha) &= 0. \label{eq:quad-c}
\end{align}
\end{proposition}

\begin{proof}
From \eqref{eq:trans1}: $a(b+c) = 8 + 4\alpha$.
Since $b + c = s - a$:
\[
a(s-a) = 8 + 4\alpha \implies as - a^2 = 8 + 4\alpha \implies a^2 - sa + (8+4\alpha) = 0.
\]
The other equations follow identically.
\end{proof}

\begin{corollary}[Quadratic Solutions]\label{cor:solutions}
By the quadratic formula:
\begin{align}
a &= \frac{s \pm \sqrt{s^2 - 4(8+4\alpha)}}{2} = \frac{s \pm \sqrt{\Delta_1}}{2}, \label{eq:sol-a}\\
b &= \frac{s \pm \sqrt{s^2 - 4(9+6\alpha)}}{2} = \frac{s \pm \sqrt{\Delta_2}}{2}, \label{eq:sol-b}\\
c &= \frac{s \pm \sqrt{s^2 - 4(5+10\alpha)}}{2} = \frac{s \pm \sqrt{\Delta_3}}{2}, \label{eq:sol-c}
\end{align}
where we define the discriminants:
\begin{align}
\Delta_1 &= s^2 - 32 - 16\alpha, \label{eq:delta1}\\
\Delta_2 &= s^2 - 36 - 24\alpha, \label{eq:delta2}\\
\Delta_3 &= s^2 - 20 - 40\alpha. \label{eq:delta3}
\end{align}
\end{corollary}

%% ============================================================
%% SECTION 4: Discriminant Constraints
%% ============================================================
\section{Discriminant Constraints}\label{sec:discriminant}

\begin{lemma}[Non-negativity Requirements]\label{lem:nonneg}
For real solutions, we require:
\begin{align}
s^2 &\geq 32 + 16\alpha \quad (\text{from } \Delta_1 \geq 0), \label{eq:req1}\\
s^2 &\geq 36 + 24\alpha \quad (\text{from } \Delta_2 \geq 0), \label{eq:req2}\\
s^2 &\geq 20 + 40\alpha \quad (\text{from } \Delta_3 \geq 0). \label{eq:req3}
\end{align}
\end{lemma}

\begin{proposition}[Binding Constraint]\label{prop:binding}
Since $\alpha = \log_2 3 \approx 1.585$:
\begin{align*}
32 + 16\alpha &\approx 32 + 25.4 = 57.4,\\
36 + 24\alpha &\approx 36 + 38.0 = 74.0,\\
20 + 40\alpha &\approx 20 + 63.4 = 83.4.
\end{align*}
The binding constraint is $s^2 \geq 20 + 40\alpha$, i.e., $s \geq \sqrt{20 + 40\alpha}$.
\end{proposition}

%% ============================================================
%% SECTION 5: Testing s = 6 + 2alpha
%% ============================================================
\section{Testing the Candidate $s = 6 + 2\alpha$}\label{sec:candidate}

We claim that $s_0 = 6 + 2\alpha = \log_2 576$ achieves the minimum.

\begin{claim}
$s_0 = 6 + 2\alpha$ satisfies the binding constraint $s^2 \geq 20 + 40\alpha$.
\end{claim}

\begin{proof}
At $s_0 = 6 + 2\alpha$:
\begin{equation}\label{eq:s0squared}
s_0^2 = (6 + 2\alpha)^2 = 36 + 24\alpha + 4\alpha^2.
\end{equation}
We need $36 + 24\alpha + 4\alpha^2 \geq 20 + 40\alpha$, i.e.,
\[
4\alpha^2 - 16\alpha + 16 \geq 0 \iff 4(\alpha - 2)^2 \geq 0.
\]
This holds for all $\alpha$, with equality iff $\alpha = 2$.
Since $\alpha = \log_2 3 < \log_2 4 = 2$, the inequality is strict.
\end{proof}

%% ============================================================
%% SECTION 6: Discriminant Perfect Square Factorizations
%% ============================================================
\section{Discriminant Perfect Square Factorizations}\label{sec:perfectsquare}

\subsection{Computation of $\Delta_1$ at $s = 6 + 2\alpha$}

\begin{lemma}[$\Delta_1$ Factorization]\label{lem:delta1}
At $s_0 = 6 + 2\alpha$:
\begin{equation}
\Delta_1 = 4(1 + \alpha)^2, \quad \sqrt{\Delta_1} = 2 + 2\alpha.
\end{equation}
\end{lemma}

\begin{proof}
\textbf{Step 1 (Substitution):}
From \eqref{eq:s0squared} and \eqref{eq:delta1}:
\[
\Delta_1 = s_0^2 - 32 - 16\alpha = (36 + 24\alpha + 4\alpha^2) - 32 - 16\alpha.
\]

\textbf{Step 2 (Grouping):}
\begin{align*}
\Delta_1 &= (36 - 32) + (24\alpha - 16\alpha) + 4\alpha^2\\
&= 4 + 8\alpha + 4\alpha^2.
\end{align*}

\textbf{Step 3 (Factorization):}
\[
4 + 8\alpha + 4\alpha^2 = 4(1 + 2\alpha + \alpha^2) = 4(1 + \alpha)^2.
\]

\textbf{Step 4 (Square Root):}
Since $\alpha = \log_2 3 > 0$, we have $1 + \alpha > 1 > 0$. Thus:
\[
\sqrt{\Delta_1} = \sqrt{4(1+\alpha)^2} = 2|1+\alpha| = 2(1+\alpha) = 2 + 2\alpha.
\]
\end{proof}

\subsection{Computation of $\Delta_2$ at $s = 6 + 2\alpha$}

\begin{lemma}[$\Delta_2$ Factorization]\label{lem:delta2}
At $s_0 = 6 + 2\alpha$:
\begin{equation}
\Delta_2 = 4\alpha^2, \quad \sqrt{\Delta_2} = 2\alpha.
\end{equation}
\end{lemma}

\begin{proof}
\textbf{Step 1 (Substitution):}
\[
\Delta_2 = s_0^2 - 36 - 24\alpha = (36 + 24\alpha + 4\alpha^2) - 36 - 24\alpha.
\]

\textbf{Step 2 (Collection):}
\[
\Delta_2 = (36 - 36) + (24\alpha - 24\alpha) + 4\alpha^2 = 0 + 0 + 4\alpha^2 = 4\alpha^2.
\]

\textbf{Step 3 (Perfect Square):}
\[
4\alpha^2 = (2\alpha)^2.
\]

\textbf{Step 4 (Square Root):}
Since $\alpha = \log_2 3 > 0$:
\[
\sqrt{\Delta_2} = \sqrt{4\alpha^2} = 2|\alpha| = 2\alpha.
\]
\end{proof}

\subsection{Computation of $\Delta_3$ at $s = 6 + 2\alpha$}

\begin{lemma}[$\Delta_3$ Factorization]\label{lem:delta3}
At $s_0 = 6 + 2\alpha$:
\begin{equation}
\Delta_3 = 4(2 - \alpha)^2, \quad \sqrt{\Delta_3} = 4 - 2\alpha.
\end{equation}
\end{lemma}

\begin{proof}
\textbf{Step 1 (Substitution):}
\[
\Delta_3 = s_0^2 - 20 - 40\alpha = (36 + 24\alpha + 4\alpha^2) - 20 - 40\alpha.
\]

\textbf{Step 2 (Collection):}
\[
\Delta_3 = (36 - 20) + (24\alpha - 40\alpha) + 4\alpha^2 = 16 - 16\alpha + 4\alpha^2.
\]

\textbf{Step 3 (Factorization):}
\[
16 - 16\alpha + 4\alpha^2 = 4(4 - 4\alpha + \alpha^2) = 4(2 - \alpha)^2.
\]
\textit{Verification:} $(2 - \alpha)^2 = 4 - 4\alpha + \alpha^2$. \checkmark

\textbf{Step 4 (Sign Analysis):}
Since $\alpha = \log_2 3$ and $3 < 4 = 2^2$, we have $\log_2 3 < \log_2 4 = 2$.
Thus $\alpha < 2$, so $2 - \alpha > 0$.

\textbf{Step 5 (Square Root):}
\[
\sqrt{\Delta_3} = \sqrt{4(2-\alpha)^2} = 2|2-\alpha| = 2(2-\alpha) = 4 - 2\alpha.
\]
\end{proof}

%% ============================================================
%% SECTION 7: The Sum Constraint
%% ============================================================
\section{The Sum Constraint}\label{sec:sumconstraint}

\begin{proposition}[Sum Constraint]\label{prop:sumconstraint}
The constraint $a + b + c = s$ requires:
\begin{equation}\label{eq:signconstraint}
\epsilon_1\sqrt{\Delta_1} + \epsilon_2\sqrt{\Delta_2} + \epsilon_3\sqrt{\Delta_3} = -s
\end{equation}
for some $\epsilon_i \in \{+1, -1\}$.
\end{proposition}

\begin{proof}
From \eqref{eq:sol-a}--\eqref{eq:sol-c}:
\[
a + b + c = \frac{3s + \epsilon_1\sqrt{\Delta_1} + \epsilon_2\sqrt{\Delta_2} + \epsilon_3\sqrt{\Delta_3}}{2} = s.
\]
Multiplying by 2:
\[
3s + \epsilon_1\sqrt{\Delta_1} + \epsilon_2\sqrt{\Delta_2} + \epsilon_3\sqrt{\Delta_3} = 2s.
\]
Rearranging:
\[
\epsilon_1\sqrt{\Delta_1} + \epsilon_2\sqrt{\Delta_2} + \epsilon_3\sqrt{\Delta_3} = -s.
\]
\end{proof}

\begin{remark}
Since $s > 0$ (as $x, y, z > 0$), the right-hand side is negative.
Therefore, at least one $\epsilon_i$ must be $-1$.
\end{remark}

%% ============================================================
%% SECTION 8: Exhaustive Sign Combination Analysis
%% ============================================================
\section{Exhaustive Sign Combination Analysis}\label{sec:signanalysis}

At $s_0 = 6 + 2\alpha$, we have computed:
\[
\sqrt{\Delta_1} = 2 + 2\alpha, \quad \sqrt{\Delta_2} = 2\alpha, \quad \sqrt{\Delta_3} = 4 - 2\alpha.
\]
The target is:
\[
-s_0 = -(6 + 2\alpha) = -6 - 2\alpha.
\]

We analyze all $2^3 = 8$ sign combinations $(\epsilon_1, \epsilon_2, \epsilon_3) \in \{+1, -1\}^3$.

\subsection{Summary Table}

\begin{center}
\begin{longtable}{|c|c|c|c|l|}
\hline
\textbf{Case} & $(\epsilon_1, \epsilon_2, \epsilon_3)$ & \textbf{LHS Expression} & \textbf{LHS Value} & \textbf{Verdict} \\
\hline
\endfirsthead
\hline
\textbf{Case} & $(\epsilon_1, \epsilon_2, \epsilon_3)$ & \textbf{LHS Expression} & \textbf{LHS Value} & \textbf{Verdict} \\
\hline
\endhead
1 & $(+,+,+)$ & $(2+2\alpha) + 2\alpha + (4-2\alpha)$ & $6 + 2\alpha$ & Impossible: LHS $> 0$ \\
\hline
2 & $(+,+,-)$ & $(2+2\alpha) + 2\alpha - (4-2\alpha)$ & $-2 + 6\alpha$ & Req.~$\alpha = -\frac{1}{2}$ \\
\hline
3 & $(+,-,+)$ & $(2+2\alpha) - 2\alpha + (4-2\alpha)$ & $6 - 2\alpha$ & Contradiction \\
\hline
4 & $(+,-,-)$ & $(2+2\alpha) - 2\alpha - (4-2\alpha)$ & $-2 + 2\alpha$ & Req.~$\alpha = -1$ \\
\hline
5 & $(-,+,+)$ & $-(2+2\alpha) + 2\alpha + (4-2\alpha)$ & $2 - 2\alpha$ & Contradiction \\
\hline
6 & $(-,+,-)$ & $-(2+2\alpha) + 2\alpha - (4-2\alpha)$ & $-6 + 2\alpha$ & Req.~$\alpha = 0$ \\
\hline
7 & $(-,-,+)$ & $-(2+2\alpha) - 2\alpha + (4-2\alpha)$ & $2 - 6\alpha$ & Req.~$\alpha = 2$ \\
\hline
8 & $(-,-,-)$ & $-(2+2\alpha) - 2\alpha - (4-2\alpha)$ & $-6 - 2\alpha$ & \textbf{VALID} \\
\hline
\end{longtable}
\end{center}

\subsection{Case-by-Case Analysis}

\subsubsection{Case 1: $(\epsilon_1, \epsilon_2, \epsilon_3) = (+1, +1, +1)$}
\begin{align*}
\text{LHS} &= (2+2\alpha) + 2\alpha + (4-2\alpha)\\
&= 2 + 2\alpha + 2\alpha + 4 - 2\alpha\\
&= 6 + 2\alpha > 0.
\end{align*}
\textbf{Verdict:} Impossible since target $= -6 - 2\alpha < 0$.

\subsubsection{Case 2: $(\epsilon_1, \epsilon_2, \epsilon_3) = (+1, +1, -1)$}
\begin{align*}
\text{LHS} &= (2+2\alpha) + 2\alpha - (4-2\alpha)\\
&= 2 + 2\alpha + 2\alpha - 4 + 2\alpha\\
&= -2 + 6\alpha.
\end{align*}
Setting LHS $= -6 - 2\alpha$:
\[
-2 + 6\alpha = -6 - 2\alpha \implies 8\alpha = -4 \implies \alpha = -\frac{1}{2}.
\]
\textbf{Verdict:} Impossible since $\alpha = \log_2 3 > 0 \not\in \{-\frac{1}{2}\}$.

\subsubsection{Case 3: $(\epsilon_1, \epsilon_2, \epsilon_3) = (+1, -1, +1)$}
\begin{align*}
\text{LHS} &= (2+2\alpha) - 2\alpha + (4-2\alpha)\\
&= 2 + 2\alpha - 2\alpha + 4 - 2\alpha\\
&= 6 - 2\alpha.
\end{align*}
Setting LHS $= -6 - 2\alpha$:
\[
6 - 2\alpha = -6 - 2\alpha \implies 6 = -6.
\]
\textbf{Verdict:} Impossible (algebraic contradiction).

\subsubsection{Case 4: $(\epsilon_1, \epsilon_2, \epsilon_3) = (+1, -1, -1)$}
\begin{align*}
\text{LHS} &= (2+2\alpha) - 2\alpha - (4-2\alpha)\\
&= 2 + 2\alpha - 2\alpha - 4 + 2\alpha\\
&= -2 + 2\alpha.
\end{align*}
Setting LHS $= -6 - 2\alpha$:
\[
-2 + 2\alpha = -6 - 2\alpha \implies 4\alpha = -4 \implies \alpha = -1.
\]
\textbf{Verdict:} Impossible since $\alpha > 0$.

\subsubsection{Case 5: $(\epsilon_1, \epsilon_2, \epsilon_3) = (-1, +1, +1)$}
\begin{align*}
\text{LHS} &= -(2+2\alpha) + 2\alpha + (4-2\alpha)\\
&= -2 - 2\alpha + 2\alpha + 4 - 2\alpha\\
&= 2 - 2\alpha.
\end{align*}
Setting LHS $= -6 - 2\alpha$:
\[
2 - 2\alpha = -6 - 2\alpha \implies 2 = -6.
\]
\textbf{Verdict:} Impossible (algebraic contradiction).

\subsubsection{Case 6: $(\epsilon_1, \epsilon_2, \epsilon_3) = (-1, +1, -1)$}
\begin{align*}
\text{LHS} &= -(2+2\alpha) + 2\alpha - (4-2\alpha)\\
&= -2 - 2\alpha + 2\alpha - 4 + 2\alpha\\
&= -6 + 2\alpha.
\end{align*}
Setting LHS $= -6 - 2\alpha$:
\[
-6 + 2\alpha = -6 - 2\alpha \implies 4\alpha = 0 \implies \alpha = 0.
\]
\textbf{Verdict:} Impossible since $\alpha = \log_2 3 > 0$ (boundary excluded).

\subsubsection{Case 7: $(\epsilon_1, \epsilon_2, \epsilon_3) = (-1, -1, +1)$}
\begin{align*}
\text{LHS} &= -(2+2\alpha) - 2\alpha + (4-2\alpha)\\
&= -2 - 2\alpha - 2\alpha + 4 - 2\alpha\\
&= 2 - 6\alpha.
\end{align*}
Setting LHS $= -6 - 2\alpha$:
\[
2 - 6\alpha = -6 - 2\alpha \implies -4\alpha = -8 \implies \alpha = 2.
\]
\textbf{Verdict:} Impossible since $\alpha = \log_2 3 < 2$ (boundary excluded).

\subsubsection{Case 8: $(\epsilon_1, \epsilon_2, \epsilon_3) = (-1, -1, -1)$}

This is the key case. We compute:
\begin{align*}
\text{LHS} &= -(2+2\alpha) - 2\alpha - (4-2\alpha)\\
&= (-2 - 2\alpha) + (-2\alpha) + (-4 + 2\alpha).
\end{align*}

\textbf{Step-by-step:}
\begin{enumerate}
\item Distribute signs: $(-1)(2+2\alpha) = -2 - 2\alpha$
\item Distribute signs: $(-1)(2\alpha) = -2\alpha$
\item Distribute signs: $(-1)(4-2\alpha) = -4 + 2\alpha$
\item Group constants: $(-2) + (-4) = -6$
\item Group $\alpha$ terms: $(-2\alpha) + (-2\alpha) + (2\alpha) = -2\alpha$
\item Combine: $\text{LHS} = -6 - 2\alpha$
\end{enumerate}

\textbf{Verification:}
\[
\text{LHS} = -6 - 2\alpha = -(6 + 2\alpha) = -s_0. \quad \checkmark
\]

\textbf{Verdict:} This is the unique valid solution!

\begin{theorem}[Unique Sign Combination]\label{thm:uniquesign}
By exhaustive enumeration of all 8 sign combinations, only $(\epsilon_1, \epsilon_2, \epsilon_3) = (-1, -1, -1)$ yields LHS $= -s$ for $\alpha \in (0, 2)$.
\end{theorem}

%% ============================================================
%% SECTION 9: Solution Values
%% ============================================================
\section{Explicit Solution Values}\label{sec:solution}

\begin{proposition}[Values of $a$, $b$, $c$]\label{prop:abc}
With $\epsilon_1 = \epsilon_2 = \epsilon_3 = -1$ and $s = 6 + 2\alpha$:
\begin{align}
a &= \frac{(6+2\alpha) - (2+2\alpha)}{2} = \frac{4}{2} = 2, \label{eq:aval}\\
b &= \frac{(6+2\alpha) - 2\alpha}{2} = \frac{6}{2} = 3, \label{eq:bval}\\
c &= \frac{(6+2\alpha) - (4-2\alpha)}{2} = \frac{2+4\alpha}{2} = 1 + 2\alpha. \label{eq:cval}
\end{align}
\end{proposition}

\begin{proof}
Using \eqref{eq:sol-a}--\eqref{eq:sol-c} with the minus signs:
\begin{align*}
a &= \frac{s - \sqrt{\Delta_1}}{2} = \frac{(6+2\alpha) - (2+2\alpha)}{2} = \frac{4}{2} = 2.\\
b &= \frac{s - \sqrt{\Delta_2}}{2} = \frac{(6+2\alpha) - 2\alpha}{2} = \frac{6}{2} = 3.\\
c &= \frac{s - \sqrt{\Delta_3}}{2} = \frac{(6+2\alpha) - (4-2\alpha)}{2} = \frac{2+4\alpha}{2} = 1 + 2\alpha.
\end{align*}
\end{proof}

\begin{corollary}[Values of $x$, $y$, $z$]\label{cor:xyz}
\begin{align}
x &= 2^a = 2^2 = 4, \label{eq:xval}\\
y &= 2^b = 2^3 = 8, \label{eq:yval}\\
z &= 2^c = 2^{1+2\alpha} = 2 \cdot 2^{2\alpha} = 2 \cdot (2^{\alpha})^2 = 2 \cdot 3^2 = 18. \label{eq:zval}
\end{align}
\end{corollary}

\begin{proposition}[Verification of Sum]\label{prop:verifysum}
\[
a + b + c = 2 + 3 + (1 + 2\alpha) = 6 + 2\alpha = s_0. \quad \checkmark
\]
\end{proposition}

%% ============================================================
%% SECTION 10: Verification of Original Constraints
%% ============================================================
\section{Verification of Original Constraints}\label{sec:verify}

We verify that $(x, y, z) = (4, 8, 18)$ satisfies all three original constraints.

\subsection{Verification of Constraint 1}

\begin{claim}
$x^{\log_2(yz)} = 2^8 \cdot 3^4$.
\end{claim}

\begin{proof}
\begin{align*}
yz &= 8 \cdot 18 = 144 = 16 \cdot 9 = 2^4 \cdot 3^2.\\
\log_2(yz) &= \log_2(2^4 \cdot 3^2) = 4 + 2\alpha.\\
x^{\log_2(yz)} &= 4^{4+2\alpha} = (2^2)^{4+2\alpha} = 2^{8+4\alpha}.\\
\intertext{Since $2^\alpha = 3$:}
2^{8+4\alpha} &= 2^8 \cdot 2^{4\alpha} = 2^8 \cdot (2^\alpha)^4 = 2^8 \cdot 3^4. \quad \checkmark
\end{align*}
\end{proof}

\subsection{Verification of Constraint 2}

\begin{claim}
$y^{\log_2(zx)} = 2^9 \cdot 3^6$.
\end{claim}

\begin{proof}
\begin{align*}
zx &= 18 \cdot 4 = 72 = 8 \cdot 9 = 2^3 \cdot 3^2.\\
\log_2(zx) &= \log_2(2^3 \cdot 3^2) = 3 + 2\alpha.\\
y^{\log_2(zx)} &= 8^{3+2\alpha} = (2^3)^{3+2\alpha} = 2^{9+6\alpha}.\\
2^{9+6\alpha} &= 2^9 \cdot 2^{6\alpha} = 2^9 \cdot (2^\alpha)^6 = 2^9 \cdot 3^6. \quad \checkmark
\end{align*}
\end{proof}

\subsection{Verification of Constraint 3}

\begin{claim}
$z^{\log_2(xy)} = 2^5 \cdot 3^{10}$.
\end{claim}

\begin{proof}
\begin{align*}
xy &= 4 \cdot 8 = 32 = 2^5.\\
\log_2(xy) &= \log_2(2^5) = 5.\\
z^{\log_2(xy)} &= 18^5 = (2 \cdot 3^2)^5 = 2^5 \cdot 3^{10}. \quad \checkmark
\end{align*}
\end{proof}

%% ============================================================
%% SECTION 11: Minimality Proof
%% ============================================================
\section{Proof of Minimality}\label{sec:minimality}

We now prove that $s_0 = 6 + 2\alpha$ is indeed the minimum value of $s = a + b + c$.

\subsection{Definition of the Constraint Function}

\begin{definition}\label{def:f}
For $s \geq \sqrt{20 + 40\alpha}$, define:
\begin{equation}\label{eq:f}
f(s) = \sqrt{s^2 - 32 - 16\alpha} + \sqrt{s^2 - 36 - 24\alpha} + \sqrt{s^2 - 20 - 40\alpha} - s.
\end{equation}
\end{definition}

The constraint equation (for the all-minus case) is $f(s) = 0$.

\subsection{Verification that $f(s_0) = 0$}

\begin{lemma}\label{lem:f-zero}
$f(s_0) = 0$ where $s_0 = 6 + 2\alpha$.
\end{lemma}

\begin{proof}
Substituting the computed discriminant roots:
\begin{align*}
f(s_0) &= \sqrt{\Delta_1} + \sqrt{\Delta_2} + \sqrt{\Delta_3} - s_0\\
&= (2 + 2\alpha) + 2\alpha + (4 - 2\alpha) - (6 + 2\alpha).
\end{align*}
Simplifying the first three terms:
\[
(2 + 2\alpha) + 2\alpha + (4 - 2\alpha) = 2 + 2\alpha + 2\alpha + 4 - 2\alpha = 6 + 2\alpha.
\]
Therefore:
\[
f(s_0) = (6 + 2\alpha) - (6 + 2\alpha) = 0. \quad \checkmark
\]
\end{proof}

\subsection{Strict Monotonicity of $f$}

\begin{lemma}\label{lem:fprime}
$f'(s) > 0$ for all $s$ in the domain.
\end{lemma}

\begin{proof}
Write $f(s) = \sqrt{s^2 - c_1} + \sqrt{s^2 - c_2} + \sqrt{s^2 - c_3} - s$ where:
\[
c_1 = 32 + 16\alpha, \quad c_2 = 36 + 24\alpha, \quad c_3 = 20 + 40\alpha.
\]

By the chain rule:
\[
\frac{d}{ds}\sqrt{s^2 - c} = \frac{1}{2\sqrt{s^2 - c}} \cdot 2s = \frac{s}{\sqrt{s^2 - c}}.
\]

Therefore:
\begin{equation}\label{eq:fprime}
f'(s) = \frac{s}{\sqrt{s^2 - c_1}} + \frac{s}{\sqrt{s^2 - c_2}} + \frac{s}{\sqrt{s^2 - c_3}} - 1.
\end{equation}

\textbf{Claim:} Each fraction $\frac{s}{\sqrt{s^2 - c_i}} > 1$.

\textbf{Proof of claim:}
For $\frac{s}{\sqrt{s^2 - c_i}} > 1$, we need $s > \sqrt{s^2 - c_i}$, i.e., $s^2 > s^2 - c_i$, i.e., $c_i > 0$.

Since $\alpha > 0$:
\begin{align*}
c_1 &= 32 + 16\alpha > 32 > 0,\\
c_2 &= 36 + 24\alpha > 36 > 0,\\
c_3 &= 20 + 40\alpha > 20 > 0.
\end{align*}

Therefore, for $s > 0$ in the domain:
\[
f'(s) > 1 + 1 + 1 - 1 = 2 > 0.
\]
\end{proof}

\subsection{Uniqueness of the Minimum}

\begin{theorem}[Uniqueness]\label{thm:uniqueness}
$s_0 = 6 + 2\alpha$ is the unique solution to $f(s) = 0$.
\end{theorem}

\begin{proof}
\begin{enumerate}
\item The domain of $f$ is $[s_{\min}, \infty)$ where $s_{\min} = \sqrt{20 + 40\alpha}$ ensures all radicands are non-negative.
\item $f$ is continuous on its domain as a composition of continuous functions.
\item By Lemma~\ref{lem:fprime}, $f'(s) > 0$ for all $s$ in the domain.
\item By the Mean Value Theorem, continuous $f$ with $f'(s) > 0$ everywhere implies $f$ is strictly increasing.
\item For strictly increasing $f$: if $f(s_0) = 0$, then:
\begin{itemize}
\item $f(s) < 0$ for $s < s_0$, and
\item $f(s) > 0$ for $s > s_0$.
\end{itemize}
\item By Lemma~\ref{lem:f-zero}, $f(s_0) = 0$ at $s_0 = 6 + 2\alpha$.
\end{enumerate}
Therefore $s_0 = 6 + 2\alpha$ is the unique zero of $f$, and no solution exists for $s < s_0$.
\end{proof}

\begin{corollary}[Minimum Product]\label{cor:minproduct}
$s = a + b + c \geq s_0 = 6 + 2\alpha$, with equality achieved.
Therefore:
\[
xyz = 2^s \geq 2^{6+2\alpha} = 2^6 \cdot 2^{2\alpha} = 64 \cdot (2^\alpha)^2 = 64 \cdot 9 = 576.
\]
\end{corollary}

%% ============================================================
%% SECTION 12: Conclusion
%% ============================================================
\section{Conclusion}\label{sec:conclusion}

\begin{proof}[Proof of Theorem~\ref{thm:main}]
We have shown:
\begin{enumerate}
\item The constraints transform to a quadratic system in logarithmic variables.
\item The system admits real solutions only when $s = a + b + c \geq \sqrt{20 + 40\alpha}$.
\item At $s_0 = 6 + 2\alpha$, all discriminants are perfect squares.
\item Exhaustive analysis of all 8 sign combinations shows only $(-1, -1, -1)$ is valid.
\item The constraint function $f(s) = 0$ has a unique solution at $s_0 = 6 + 2\alpha$ by strict monotonicity.
\item The explicit solution $(x, y, z) = (4, 8, 18)$ satisfies all original constraints.
\item Therefore, $xyz = 4 \cdot 8 \cdot 18 = 576$ is the minimum.
\end{enumerate}
\end{proof}

\begin{equation}
\boxed{xyz_{\min} = 576}
\end{equation}

\appendix

\section{Numerical Verification}\label{app:numerical}

For reference, $\alpha = \log_2 3 \approx 1.5849625007211563$.

\begin{align*}
s_0 &= 6 + 2\alpha \approx 9.1699250014423126\\
2^{s_0} &= 2^{6 + 2\log_2 3} = 2^6 \cdot 3^2 = 64 \cdot 9 = 576 \quad \checkmark
\end{align*}

\section{Graph Metadata}\label{app:metadata}

This proof was formalized using the Alethfeld proof system from graph \texttt{graph-51acde-43ac9a}, containing 40 nodes at depth 0--1, with 67 expansion nodes at depths 2--3.

Key verified nodes:
\begin{itemize}
\item \texttt{:1-000001} -- Logarithmic transformation
\item \texttt{:1-000011, :1-000012, :1-000013} -- Discriminant factorizations
\item \texttt{:1-000015} -- Valid sign combination verification
\item \texttt{:1-000034} -- Exhaustive sign analysis (8 cases)
\item \texttt{:1-000031, :1-000032, :1-000033} -- Minimality via monotonicity
\end{itemize}

\end{document}
