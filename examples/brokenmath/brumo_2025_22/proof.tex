\documentclass[11pt]{article}
\usepackage{amsmath,amssymb,amsthm}
\usepackage{enumitem}
\usepackage{geometry}
\geometry{margin=1in}

\newtheorem{theorem}{Theorem}
\newtheorem{lemma}[theorem]{Lemma}
\newtheorem{claim}[theorem]{Claim}
\newtheorem{definition}[theorem]{Definition}

\newenvironment{proofsteps}{\begin{enumerate}[label=\textbf{Step \arabic*.},leftmargin=*]}{\end{enumerate}}
\newcommand{\by}[1]{\hfill\textit{(#1)}}

\title{Semi-Magic Squares: Counting Arrangements\\
\large Alethfeld Proof --- Graph: graph-a8cc11-ee90f2 v45}
\author{}
\date{}

\begin{document}

\maketitle

\begin{theorem}[brumo\_2025\_22]
Digits $1$ through $9$ are placed on a $3 \times 3$ square such that all rows and columns sum to the same value. Diagonals do not need to sum to the same value. There are exactly $36$ ways to do this (counting arrangements up to transpose symmetry).
\end{theorem}

\begin{proof}
We proceed in several steps.

\begin{proofsteps}

\item \label{step:def001} \textbf{Definition.}
A \emph{valid arrangement} is a bijection $\sigma: \{1,\ldots,9\} \to \{(i,j) : 1 \le i,j \le 3\}$ such that there exists a constant $S$ where $\sum_{j=1}^{3} \sigma^{-1}(i,j) = S$ for all rows $i \in \{1,2,3\}$ and $\sum_{i=1}^{3} \sigma^{-1}(i,j) = S$ for all columns $j \in \{1,2,3\}$.
\by{definition}

\item \label{step:clm001} \textbf{The magic constant.}
For any valid arrangement, the common sum $S = 15$.

\textit{Proof:} The sum of all entries is $1 + 2 + \cdots + 9 = 45$. Since all three rows sum to $S$, we have $3S = 45$, giving $S = 15$.
\by{algebraic}

\item \label{step:clm010} \textbf{Constraint formulation.}
Label the grid positions as: top row $(a,b,c)$, middle row $(d,e,f)$, bottom row $(g,h,i)$. The constraint equations are:
\begin{align*}
a+b+c &= d+e+f = g+h+i = 15 \quad\text{(rows)}\\
a+d+g &= b+e+h = c+f+i = 15 \quad\text{(columns)}
\end{align*}
From row 2 and column 2: $d+f = 15-e$ and $b+h = 15-e$.
\by{algebraic rewrite}

\item \label{step:clm016} \textbf{Pair structure.}
For center $e \in \{1,\ldots,9\}$, define $R_e$ as the set of unordered pairs $\{a,b\}$ with $a,b \in \{1,\ldots,9\} \setminus \{e\}$, $a \ne b$, and $a + b = 15 - e$.

\textit{Key observation:} The elements of $R_e$ are pairwise disjoint (each value in $\{1,\ldots,9\} \setminus \{e\}$ appears in at most one pair).
\by{algebraic}

\item \label{step:clm005} \textbf{Pair counts.}
Computing $|R_e|$ for each $e$:
\begin{itemize}
\item For $e=5$: pairs summing to 10 are $\{1,9\},\{2,8\},\{3,7\},\{4,6\}$, so $|R_5|=4$
\item For $e \in \{2,4,6,8\}$: $|R_e|=3$
\item For $e \in \{1,3,7,9\}$: $|R_e|=2$
\end{itemize}
\by{enumeration}

\item \label{step:clm018} \textbf{Compatibility constraint.}
Not every choice of disjoint pairs $P_1, P_2 \in R_e$ yields a valid arrangement. The corners $C = \{1,\ldots,9\} \setminus (\{e\} \cup P_1 \cup P_2)$ must satisfy additional compatibility conditions that restrict which pair choices work.
\by{case analysis}

\item \label{step:clm019} \textbf{Exhaustive verification.}
By exhaustive verification: for each center $e \in \{1,\ldots,9\}$, exactly one compatible pair choice $(P_1, P_2)$ exists (up to swapping $P_1 \leftrightarrow P_2$, which gives the transpose). For each such choice, fixing the assignment $P_1 \to \text{row}$ and $P_2 \to \text{column}$, there are exactly 4 valid configurations ($2$ orderings for $P_1 \times 2$ orderings for $P_2$, each yielding a unique valid corner assignment).
\by{case split}

\item \label{step:clm013} \textbf{Total enumeration.}
By exhaustive enumeration over all $9! = 362880$ permutations, checking the 6 linear constraints (3 row sums and 3 column sums all equal to 15), we find exactly 72 valid grid configurations.
\by{computation}

\item \label{step:clm014} \textbf{Transpose symmetry.}
The transpose operation (swapping rows and columns) preserves the property of having all row and column sums equal to 15. For any valid configuration $G$, its transpose $G^T$ is also valid. Furthermore, $G \ne G^T$ for all 72 configurations (no configuration equals its own transpose).
\by{algebraic}

\item \label{step:clm020} \textbf{Final count.}
The total count is: 9 centers $\times$ 4 configurations per center (with row/column assignment fixed) $= 36$ configurations. Equivalently, there are 72 labeled arrangements, and the transpose operation partitions these into 36 pairs (no arrangement equals its transpose). The count 36 represents arrangements up to the symmetry of swapping rows with columns.
\by{algebraic}

\end{proofsteps}

Therefore, there are exactly 36 ways to place digits 1 through 9 on a $3 \times 3$ square such that all rows and columns sum to the same value (counting arrangements up to transpose symmetry). \qed
\end{proof}

\section*{Appendix: Sample Valid Configurations}

For center $e = 5$ with pairs $\{1,9\}$ and $\{3,7\}$, one valid configuration is:
\[
\begin{pmatrix}
8 & 3 & 4 \\
1 & 5 & 9 \\
6 & 7 & 2
\end{pmatrix}
\]
Verification: Rows sum to $8+3+4=15$, $1+5+9=15$, $6+7+2=15$. Columns sum to $8+1+6=15$, $3+5+7=15$, $4+9+2=15$.

\end{document}
