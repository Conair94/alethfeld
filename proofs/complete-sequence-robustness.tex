\documentclass[11pt,a4paper]{article}

% Packages
\usepackage{amsmath,amssymb,amsthm}
\usepackage{mathtools}
\usepackage{enumitem}
\usepackage[margin=1in]{geometry}
\usepackage{hyperref}
\usepackage{cleveref}

% Theorem environments
\newtheorem{theorem}{Theorem}
\newtheorem{lemma}[theorem]{Lemma}
\newtheorem{proposition}[theorem]{Proposition}
\newtheorem{corollary}[theorem]{Corollary}
\theoremstyle{definition}
\newtheorem{definition}[theorem]{Definition}
\newtheorem{construction}[theorem]{Construction}
\theoremstyle{remark}
\newtheorem{remark}[theorem]{Remark}
\newtheorem{clarification}[theorem]{Clarification}

% Lamport-style proof steps
\newcounter{proofstep}
\newcounter{proofsubstep}[proofstep]
\newcounter{proofsubsubstep}[proofsubstep]
\newcounter{proofsubsubsubstep}[proofsubsubstep]

\newenvironment{proofsteps}{%
  \setcounter{proofstep}{0}%
  \begin{list}{\textbf{\arabic{proofstep}.}}{%
    \usecounter{proofstep}%
    \setlength{\leftmargin}{2em}%
    \setlength{\itemsep}{0.5em}%
  }%
}{\end{list}}

\newenvironment{proofsubsteps}{%
  \begin{list}{\textbf{\arabic{proofstep}.\arabic{proofsubstep}.}}{%
    \usecounter{proofsubstep}%
    \setlength{\leftmargin}{2em}%
    \setlength{\itemsep}{0.3em}%
  }%
}{\end{list}}

\newenvironment{proofsubsubsteps}{%
  \begin{list}{\textbf{\arabic{proofstep}.\arabic{proofsubstep}.\arabic{proofsubsubstep}.}}{%
    \usecounter{proofsubsubstep}%
    \setlength{\leftmargin}{2em}%
    \setlength{\itemsep}{0.2em}%
  }%
}{\end{list}}

\newenvironment{proofsubsubsubsteps}{%
  \begin{list}{\textbf{\arabic{proofstep}.\arabic{proofsubstep}.\arabic{proofsubsubstep}.\arabic{proofsubsubsubstep}.}}{%
    \usecounter{proofsubsubsubstep}%
    \setlength{\leftmargin}{2em}%
    \setlength{\itemsep}{0.2em}%
  }%
}{\end{list}}

% Custom commands
\newcommand{\by}[1]{\hfill\textit{[#1]}}
\newcommand{\Am}{A_m}

\title{Complete Sequence Robustness Theorem}
\author{Alethfeld Proof System\\[0.5em]\small Graph ID: \texttt{graph-8a76a3-245578}, Version 164}
\date{December 2025}

\begin{document}
\maketitle

\begin{abstract}
We prove that for all integers $0 \leq m < n$, there exists a complete sequence of positive integers that remains complete after removing any $m$ elements, but becomes incomplete after removing some $n$ elements. The proof is constructive, exhibiting a family of sequences $\{A_m\}_{m \geq 0}$ where each power of 2 appears with multiplicity $(m+1)$.
\end{abstract}

\tableofcontents
\newpage

%=============================================================================
\section{Preliminaries}
%=============================================================================

\begin{clarification}[Multiset Conventions]
\label{clar:multiset}
Throughout this proof, $A$ denotes a non-decreasing sequence (multiset with ordering) of positive integers. The notation $\{a_1 \leq a_2 \leq \cdots\}$ specifies weak ordering. ``Distinct elements'' means distinct \emph{positions} (indices), not distinct values. Multiset subtraction $A \setminus S$ removes elements by position.
\end{clarification}

\begin{definition}[Complete Sequence]
\label{def:complete}
A sequence $A = \{a_1 \leq a_2 \leq \cdots\}$ of positive integers is \textbf{complete} if every positive integer can be represented as a sum of distinct elements from $A$.
\end{definition}

\begin{definition}[$k$-Subcomplete]
\label{def:subcomplete}
A complete sequence $A$ is \textbf{$k$-subcomplete} if $A \setminus S$ remains complete for every subset $S \subseteq A$ with $|S| = k$.
\end{definition}

%=============================================================================
\section{Key Lemmas}
%=============================================================================

\begin{lemma}[Brown's Criterion \cite{brown1961}]
\label{lem:brown}
A non-decreasing sequence $A = \{a_1 \leq a_2 \leq \cdots\}$ of positive integers is complete if and only if:
\begin{enumerate}[label=(\roman*)]
    \item $a_1 = 1$, and
    \item $a_{k+1} \leq 1 + \sum_{i=1}^{k} a_i$ for all $k \geq 1$.
\end{enumerate}
\end{lemma}

\begin{remark}
This criterion is sometimes called ``Cassels' criterion'' in the literature, though the characterization theorem was published by J.~L.~Brown Jr.\ in 1961.
\end{remark}

\begin{lemma}[Superset Preservation]
\label{lem:superset}
If $A$ is a complete sequence and $A \subseteq B$ (as multisets), then $B$ is complete.
\end{lemma}

\begin{proof}
Every positive integer $n$ has a representation as a sum of distinct elements from $A$. Since $A \subseteq B$, those same elements exist in $B$, so $n$ is also representable using elements of $B$.
\end{proof}

\begin{lemma}[Powers of 2 Core]
\label{lem:powers2}
Any multiset $B$ containing at least one copy of each power of 2 (i.e., $1, 2, 4, 8, \ldots \in B$) is complete.
\end{lemma}

\begin{proof}
The sequence $P = \{1, 2, 4, 8, \ldots\}$ is complete by Brown's criterion:
\begin{proofsubsteps}
    \item For $P = \{2^0, 2^1, 2^2, \ldots\}$, we have $a_k = 2^{k-1}$, so $a_1 = 2^0 = 1$. \by{substitution}

    \item For $k \geq 1$, the sum $\sum_{i=1}^{k} a_i = \sum_{i=1}^{k} 2^{i-1} = \sum_{j=0}^{k-1} 2^j$ where $j = i-1$. \by{index substitution}

    \item The geometric series formula gives $\sum_{j=0}^{k-1} 2^j = \frac{2^k - 1}{2-1} = 2^k - 1$.
    \begin{proofsubsubsteps}
        \item \textbf{Base case} ($k=1$): $\sum_{j=0}^{0} 2^j = 2^0 = 1 = \frac{2^1-1}{1} = 1$. $\checkmark$
        \item \textbf{Inductive hypothesis}: Assume $\sum_{j=0}^{k-1} 2^j = 2^k - 1$ holds for some $k \geq 1$.
        \item \textbf{Inductive step}: $\sum_{j=0}^{k} 2^j = \sum_{j=0}^{k-1} 2^j + 2^k = (2^k - 1) + 2^k = 2 \cdot 2^k - 1 = 2^{k+1} - 1$. $\checkmark$
        \item By induction, $\sum_{j=0}^{k-1} 2^j = 2^k - 1$ for all $k \geq 1$. $\square$
    \end{proofsubsubsteps}

    \item For $k \geq 1$, we have $a_{k+1} = 2^{(k+1)-1} = 2^k$. \by{substitution}

    \item The inequality $2^k \leq 1 + (2^k - 1) = 2^k$ holds with equality for all $k \geq 1$. \by{arithmetic}
\end{proofsubsteps}
Since $B$ contains $P$ as a submultiset and $P$ is complete, $B$ is complete by \Cref{lem:superset}.
\end{proof}

%=============================================================================
\section{Main Theorem}
%=============================================================================

\begin{theorem}[Complete Sequence Robustness]
\label{thm:main}
For all integers $0 \leq m < n$, there exists a complete sequence $A = \{a_1 \leq a_2 \leq \cdots\}$ of positive integers such that $A$ remains complete after removing any $m$ elements, but there exist $n$ elements whose removal makes $A$ incomplete.
\end{theorem}

\begin{construction}
\label{con:Am}
For any $m \geq 0$, define $\Am$ to be the sequence where each power of 2 appears with multiplicity $(m+1)$:
\[
\Am = \{\underbrace{1, 1, \ldots, 1}_{m+1}, \underbrace{2, 2, \ldots, 2}_{m+1}, \underbrace{4, 4, \ldots, 4}_{m+1}, \ldots\}
\]
\end{construction}

\begin{proof}[Proof of \Cref{thm:main}]
We prove three claims about $\Am$:

\begin{proofsteps}

%-----------------------------------------------------------------------------
\item \textbf{Claim 1: $\Am$ is complete.} \label{step:complete}
%-----------------------------------------------------------------------------

We verify Brown's criterion (\Cref{lem:brown}):
\begin{proofsubsteps}
    \item The sequence $\Am$ has $a_1 = 1$, satisfying condition (i). \by{\Cref{con:Am}}

    \item For $\Am$, the first $(m+1) \cdot k$ elements are the powers $2^0, 2^1, \ldots, 2^{k-1}$, each with multiplicity $(m+1)$. Their sum is:
    \[
    (m+1) \cdot (1 + 2 + 4 + \cdots + 2^{k-1}) = (m+1) \cdot (2^k - 1)
    \]
    \by{geometric series}

    \item The next element after the first $(m+1) \cdot k$ elements is $2^k$. We verify condition (ii):
    \begin{align*}
    2^k &\leq 1 + (m+1)(2^k - 1) \\
        &= (m+1) \cdot 2^k - m
    \end{align*}
    This holds since $2^k \leq (m+1) \cdot 2^k - m$ iff $m \cdot 2^k \geq m$ iff $2^k \geq 1$, which is true for all $k \geq 0$. \by{algebra}

    \item By Brown's criterion, $\Am$ is complete. \by{modus ponens}
\end{proofsubsteps}

%-----------------------------------------------------------------------------
\item \textbf{Claim 2: $\Am$ is $m$-subcomplete.} \label{step:subcomplete}
%-----------------------------------------------------------------------------

Let $S \subseteq \Am$ be arbitrary with $|S| = m$. We prove $\Am \setminus S$ is complete:
\begin{proofsubsteps}
    \item For each power $2^j$, the set $S$ contains at most $m$ copies of $2^j$ since $|S| = m$ total. \by{cardinality}

    \item \textbf{Arithmetic}: $\Am$ has $(m+1)$ copies of each $2^j$. Since $S$ removes at most $m$ copies of any value, at least $(m+1) - m = 1$ copy of each $2^j$ remains in $\Am \setminus S$. \by{subtraction}

    \item Since at least one copy of each $2^j$ remains, the support of $\Am \setminus S$ contains $\{2^0, 2^1, 2^2, \ldots\} = \{1, 2, 4, 8, \ldots\}$. \by{step 2.2}

    \item \textbf{Powers of 2 Core} (\Cref{lem:powers2}): Any multiset containing at least one copy of each power of 2 is complete. This follows from binary representation: every positive integer $n$ has a unique binary expansion $n = \sum_{i \in I} 2^i$, so $n$ is representable using distinct powers of 2. \by{\Cref{lem:powers2}}

    \item \textbf{Superset Preservation} (\Cref{lem:superset}): Since $\{1,2,4,\ldots\} \subseteq \Am \setminus S$ and $\{1,2,4,\ldots\}$ is complete, $\Am \setminus S$ is complete. \by{\Cref{lem:superset}}
\end{proofsubsteps}
Since $S$ was arbitrary with $|S| = m$, by \Cref{def:subcomplete}, $\Am$ is $m$-subcomplete.

%-----------------------------------------------------------------------------
\item \textbf{Claim 3: $\Am$ is not $(m+1)$-subcomplete.} \label{step:notsubcomplete}
%-----------------------------------------------------------------------------

We exhibit a witness set $S^*$ with $|S^*| = m+1$ such that $\Am \setminus S^*$ is not complete:
\begin{proofsubsteps}
    \item Define $S^* = \{\text{all } (m+1) \text{ copies of } 1 \text{ in } \Am\}$. Then $|S^*| = m+1$. \by{\Cref{con:Am}}

    \item $\Am \setminus S^* = \{2, 2, \ldots, 4, 4, \ldots, 8, 8, \ldots\}$ consists of $(m+1)$ copies each of $2^k$ for $k \geq 1$. \by{set difference}

    \item The smallest element of $\Am \setminus S^*$ is $2^1 = 2$. Hence every element of $\Am \setminus S^*$ is $\geq 2$. \by{step 3.2}

    \item \textbf{Case analysis for representing 1}: Either (a) use the empty sum, or (b) use a non-empty sum of distinct elements from $\Am \setminus S^*$. \by{exhaustive cases}
    \begin{proofsubsubsteps}
        \item \textbf{Case (a)}: The empty sum equals $0 \neq 1$.
        \item \textbf{Case (b)}: Let $X \subseteq \Am \setminus S^*$ be non-empty.
        \begin{proofsubsubsubsteps}
            \item Then $X$ contains at least one element $x$. \by{non-empty}
            \item Since $x \in \Am \setminus S^*$ and every element of $\Am \setminus S^*$ is $\geq 2$ (step 3.3), we have $x \geq 2$. \by{step 3.3}
            \item The sum of elements in $X$ is $\geq x$ (since all elements are positive). By step 3.4.2.2, $x \geq 2$. Hence $\text{sum}(X) \geq 2$. \by{arithmetic}
            \item Since $X$ was arbitrary non-empty subset, any non-empty sum is $\geq 2 > 1$. \by{universal generalization}
        \end{proofsubsubsubsteps}
    \end{proofsubsubsteps}

    \item By case exhaustion: 1 cannot be represented as a sum of distinct elements from $\Am \setminus S^*$. \by{steps 3.4.1, 3.4.2}

    \item Since $\Am \setminus S^*$ cannot represent 1 (a positive integer), $\Am \setminus S^*$ is not complete. \by{\Cref{def:complete}}

    \item Since $|S^*| = m+1$ and $\Am \setminus S^*$ is not complete, $\Am$ is not $(m+1)$-subcomplete. \by{\Cref{def:subcomplete}}
\end{proofsubsteps}

%-----------------------------------------------------------------------------
\item \textbf{Claim 4: For any $n > m$, $\Am$ is not $n$-subcomplete.} \label{step:extension}
%-----------------------------------------------------------------------------

Let $n > m$ be arbitrary. We construct a witness set $T$ with $|T| = n$:
\begin{proofsubsteps}
    \item The multiset $\Am \setminus S^*$ is infinite (contains infinitely many copies of $2, 4, 8, \ldots$). Since $(n-m-1)$ is a finite non-negative integer, we can select $(n-m-1)$ elements from $\Am \setminus S^*$. \by{infinite set}

    \item Define $T = S^* \cup \{\text{these } (n-m-1) \text{ elements}\}$. Note: $T \subseteq \Am$ by construction. \by{construction}

    \item \textbf{Cardinality} (multiset arithmetic): $|T| = |S^*| + (n-m-1) = (m+1) + (n-m-1) = n$. \by{arithmetic}

    \item \textbf{Subset relation} (multiset): Since $S^* \subseteq T$, multiset subtraction gives $\Am \setminus T \subseteq \Am \setminus S^*$ (removing more leaves less or equal). \by{multiset properties}

    \item \textbf{Monotonicity of non-representability}: By step 3.5, $\Am \setminus S^*$ cannot represent 1. By step 4.4, $\Am \setminus T \subseteq \Am \setminus S^*$. Any sum from $\Am \setminus T$ is also a sum from $\Am \setminus S^*$. Hence $\Am \setminus T$ also cannot represent 1. \by{subset property}

    \item Since $\Am \setminus T$ cannot represent 1, $\Am \setminus T$ is not complete. Since $|T| = n$, the set $T$ witnesses that $\Am$ is not $n$-subcomplete. \by{\Cref{def:subcomplete}}
\end{proofsubsteps}
Since $n > m$ was arbitrary, $\forall n > m$: $\Am$ is not $n$-subcomplete.

%-----------------------------------------------------------------------------
\item \textbf{Edge case: $m = 0$.} \label{step:edge}
%-----------------------------------------------------------------------------

When $m = 0$, we have $A_0 = \{1, 2, 4, 8, \ldots\}$ (multiplicity 1). This is \emph{trivially} 0-subcomplete: the quantification ``$\forall S$ with $|S| = 0$'' has exactly one instance ($S = \emptyset$), and $A_0 \setminus \emptyset = A_0$ is complete by step 1.

\end{proofsteps}

\medskip
\noindent\textbf{Conclusion.} For all integers $0 \leq m < n$, the sequence $\Am$ satisfies:
\begin{enumerate}[label=(\arabic*)]
    \item $\Am$ is complete (by step 1 via Brown's criterion),
    \item $\Am$ is $m$-subcomplete (by step 2 via arithmetic and superset preservation),
    \item $\Am$ is not $n$-subcomplete (by step 4 via witness construction).
\end{enumerate}
This completes the proof. \qed
\end{proof}

%=============================================================================
\section{Answer to the Original Question}
%=============================================================================

\begin{corollary}
For what values of $0 \leq m < n$ is there a complete sequence $A$ such that $A$ remains complete after removing any $m$ elements, but $A$ is not complete after removing any $n$ elements?

\textbf{Answer:} Such a sequence exists for \textbf{all} pairs $(m, n)$ with $0 \leq m < n$.
\end{corollary}

%=============================================================================
\begin{thebibliography}{9}
%=============================================================================

\bibitem{brown1961}
J.~L. Brown Jr.,
\emph{Note on complete sequences of integers},
American Mathematical Monthly \textbf{68}(6) (1961), 557--560.
DOI: \href{https://doi.org/10.2307/2311990}{10.2307/2311990}

\bibitem{cassels1960}
J.~W.~S. Cassels,
\emph{On the representation of integers as the sums of distinct summands taken from a fixed set},
Acta Scientiarum Mathematicarum (Szeged) \textbf{21} (1960), 111--124.

\end{thebibliography}

\vfill
\hrule
\smallskip
\noindent\textit{Generated by Alethfeld Proof System.}\\
\noindent\textit{Graph: \texttt{graph-8a76a3-245578}, Version 164, 58 verified nodes.}

\end{document}
