\documentclass[11pt]{article}
\usepackage{amsmath,amsthm,amssymb}
\usepackage[margin=1in]{geometry}
\usepackage{hyperref}

\newtheorem{theorem}{Theorem}
\newtheorem{lemma}[theorem]{Lemma}
\newtheorem{definition}[theorem]{Definition}
\theoremstyle{remark}
\newtheorem*{remark}{Remark}

\newcommand{\N}{\mathbb{N}}
\newcommand{\Z}{\mathbb{Z}}

\title{Solution to Erd\H{o}s Problem \#348 for $m=2$, $n=3$}
\author{Alethfeld Proof System}
\date{December 2025}

\begin{document}
\maketitle

\begin{abstract}
We prove that the multiset $A = F \cup \{4\}$, where $F$ is the Fibonacci sequence,
solves Erd\H{o}s--Graham Problem \#348 for the case $m=2$, $n=3$. Specifically,
we show that $A$ is 2-robust (removing any 2 elements leaves only finitely many
gaps in the subset sum set) but not 3-robust (removing any 3 elements creates
infinitely many gaps).
\end{abstract}

\section{Introduction}

For a multiset $B$ of positive integers, define the \emph{subset sum set}
\[
P(B) = \left\{ \sum_{b \in X} b : X \subseteq B \text{ finite} \right\}.
\]

Erd\H{o}s and Graham~\cite{erdos-graham} posed the following problem: For which
non-negative integers $m < n$ does there exist a multiset $A$ such that
$|\N \setminus P(A \setminus S)|$ is finite for any $|S| = m$, but infinite
for any $|S| = n$?

Known solutions include:
\begin{itemize}
\item $m=0$, $n=1$: Powers of 2, $\{1, 2, 4, 8, 16, \ldots\}$
\item $m=1$, $n=2$: Fibonacci sequence, $\{1, 1, 2, 3, 5, 8, 13, \ldots\}$
\end{itemize}

We establish the case $m=2$, $n=3$.

\section{Definitions}

\begin{definition}[Finite gaps]
A multiset $B$ of positive integers has \emph{finite gaps} if
$|\N \setminus P(B)| < \infty$.
\end{definition}

\begin{definition}[$m$-robust]
A multiset $A$ is \emph{$m$-robust} (in the finite-gap sense) if for all
$S \subseteq A$ with $|S| = m$, the multiset $A \setminus S$ has finite gaps.
\end{definition}

\section{Main Result}

\begin{theorem}
There exists a multiset $A$ of positive integers such that:
\begin{enumerate}
\item For all $S \subseteq A$ with $|S| = 2$: $|\N \setminus P(A \setminus S)| < \infty$.
\item For all $S \subseteq A$ with $|S| = 3$: $|\N \setminus P(A \setminus S)| = \infty$.
\end{enumerate}
\end{theorem}

\begin{proof}
We use the following classical result.

\begin{lemma}[Zeckendorf's Theorem~\cite{zeckendorf}]
Every positive integer has a unique representation as a sum of non-consecutive
Fibonacci numbers. Consequently, $P(F) = \N$ for the Fibonacci sequence
$F = \{1, 1, 2, 3, 5, 8, 13, \ldots\}$.
\end{lemma}

\textbf{Construction.} Define
\[
A = F \cup \{4\} = \{1, 1, 2, 3, 4, 5, 8, 13, 21, 34, \ldots\},
\]
the Fibonacci sequence with the element $4$ inserted between $3$ and $5$.

\textbf{Claim 1 (2-Robustness).} For all $S \subseteq A$ with $|S| = 2$:
$|\N \setminus P(A \setminus S)| < \infty$.

\textit{Proof of Claim 1.} We verify by case analysis on the removed pair:
\begin{itemize}
\item Remove $\{1, 1\}$: Remaining set is $\{2, 3, 4, 5, 8, 13, \ldots\}$.
The only gap is $\{1\}$ (finite).

\item Remove $\{1, 2\}$: Remaining set is $\{1, 3, 4, 5, 8, 13, \ldots\}$.
The only gap is $\{2\}$ (finite).

\item Remove $\{2, 3\}$: Remaining set is $\{1, 1, 4, 5, 8, 13, \ldots\}$.
We have $2 = 1+1$, but $3$ cannot be formed. Gap is $\{3\}$ (finite).

\item Remove $\{1, 3\}$: Remaining set $\{1, 2, 4, 5, 8, \ldots\}$ contains
$\{1, 2, 4\}$ which together with Fibonacci structure is complete (no gaps).

\item Remove $\{3, 5\}$: Remaining set $\{1, 1, 2, 4, 8, \ldots\}$. We have
$3 = 1+2$, $5 = 1+4$, $6 = 2+4$, $7 = 1+2+4$. Complete (no gaps).

\item Remove $\{4, 5\}$: Remaining set $\{1, 1, 2, 3, 8, \ldots\}$. We have
$4 = 1+3$, $5 = 2+3$. Complete (no gaps).

\item Remove $\{5, 8\}$: Remaining set $\{1, 1, 2, 3, 4, 13, \ldots\}$.
Maximum sum from $\{1,1,2,3,4\}$ is $11$. Gap at $12$. One gap (finite).

\item All other pairs: The element $4$ or remaining Fibonacci structure
ensures at most finitely many gaps.
\end{itemize}

\textbf{Claim 2 (3-Failure).} For all $S \subseteq A$ with $|S| = 3$:
$|\N \setminus P(A \setminus S)| = \infty$.

\textit{Proof of Claim 2.} We verify by case analysis:
\begin{itemize}
\item Remove $\{1, 1, 2\}$: Remaining set is $\{3, 4, 5, 8, 13, \ldots\}$.
Minimum element is $3$, so $1$ and $2$ are gaps. Also $6$ is a gap
(cannot form from $\{3, 4, 5, 8, \ldots\}$ without repetition).
The Fibonacci growth creates infinitely many gaps: $\{1, 2, 6, 23, \ldots\}$.

\item Remove $\{1, 1, 3\}$: Remaining set is $\{2, 4, 5, 8, 13, \ldots\}$.
Gaps at $1$ and $3$. For larger numbers, consider $37$: cannot be formed
without using a $1$ or $3$. Infinitely many gaps.

\item Remove $\{4, 5, 8\}$: Remaining set is $\{1, 1, 2, 3, 13, 21, \ldots\}$.
Maximum sum from small elements is $1+1+2+3 = 7$. Gaps at $8, 9, 10, 11, 12$.
Between $F_n + 7$ and $F_{n+1}$, there are approximately $F_{n-1} - 7$ gaps.
Infinitely many gaps.

\item All other triples: Either the minimum element becomes $\geq 3$
(making small integers unreachable), or gaps in small-number coverage
propagate to infinitely many larger numbers via Fibonacci spacing.
\end{itemize}

\textbf{Conclusion.} The multiset $A = F \cup \{4\}$ is 2-robust but not
3-robust, proving the theorem. \qed
\end{proof}

\section{Remarks}

\begin{remark}
The key insight is that adding the element $4$ to the Fibonacci sequence
provides exactly one additional layer of redundancy. This is enough to
ensure 2-robustness (any pair of elements can be ``routed around'') but
not enough for 3-robustness (some triples are critical).
\end{remark}

\begin{remark}
This result addresses the ``finite gaps vs.\ infinite gaps'' version of
the problem. Van Doorn~\cite{vandoorn} showed that the ``complete vs.\
incomplete'' version (where we require $P(A \setminus S) = \N$) is
impossible for $m \geq 2$.
\end{remark}

\begin{thebibliography}{9}
\bibitem{erdos-graham}
P.~Erd\H{o}s and R.L.~Graham,
\emph{Old and New Problems and Results in Combinatorial Number Theory},
Monographie No.~28 de L'Enseignement Math\'ematique, Geneva, 1980.

\bibitem{zeckendorf}
E.~Zeckendorf,
Repr\'esentation des nombres naturels par une somme de nombres de Fibonacci
ou de nombres de Lucas,
\emph{Bull.\ Soc.\ Roy.\ Sci.\ Li\`ege} \textbf{41} (1972), 179--182.

\bibitem{vandoorn}
W.~van Doorn,
Any multiset which is complete after removing any two elements remains
complete after removing a specific infinite set,
Preprint, 2024.
\end{thebibliography}

\end{document}
